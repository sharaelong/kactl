\chapter{Geometry}

\section{Analytic Geometry}
Area $A = \sqrt{p(p-a)(p-b)(p-c)}$ when $p = (a+b+c)/2$ \\
Circumscribed circle $R = abc/4A$, inscribed circle $r = A/p$ \\ 
Middle line length $m_a = \sqrt{2b^2+2c^2-a^2}/2$ \\
Bisector line length $s_a=\sqrt{bc[1-(\frac{a}{b+c})^2]}$ \\
\begin{tabular}{|c|c|c|c|c|}
  Name & $\alpha$ & $\beta$ & $\gamma$ & \\ \hline
  R & $a^2\mathcal{A}$ & $b^2\mathcal{B}$ & $c^2\mathcal{C}$ & $\mathcal{A}=b^2+c^2-a^2$ \\
  r & $a$ & $b$ & $c$ & $\mathcal{B} = a^2 + c^2 - b^2$ \\
  G & $1$ & $1$ & $1$ & $\mathcal{C} = a^2 + b^2 - c^2$ \\
  H & $\mathcal{BC}$ & $\mathcal{CA}$ & $\mathcal{AB}$ & \\
  Excircle(A) & $-a$ & $b$ & $c$ & 
\end{tabular} \\
With side lengths $a,b,c,d$, diagonals $e, f$, diagonals angle $\theta$, area $A$ and
magic flux $F=b^2+d^2-a^2-c^2$: $4A = 2ef \cdot \sin\theta = F\tan\theta = \sqrt{4e^2f^2-F^2}$ \\
For cyclic quadrilaterals the sum of opposite angles is $180^\circ$,
$ef = ac + bd$, and $A = \sqrt{(p-a)(p-b)(p-c)(p-d)}$. \\
$HG:GO=1:2$. H of triangle made by middle point on arc of circumscribed circle is equal to inscribed circle center of original triangle. \\

\kactlimport{PointInteger.h}
\kactlimport{PointDouble.h}
%\kactlimport{Point.h}
%\kactlimport{lineDistance.h}
\kactlimport{SegmentDistance.h}
\kactlimport{SegmentIntersection.h}
\kactlimport{AngleSort.h}
%\kactlimport{lineIntersection.h}
%\kactlimport{sideOf.h}
%\kactlimport{OnSegment.h}
%\kactlimport{linearTransformation.h}
% \kactlimport{LineProjectionReflection.h}
%\kactlimport{Angle.h}
\kactlimport{ShamosHoey.h}
\kactlimport{HalfPlaneIntersection.h}
\kactlimport{FastDelaunay.h}
\kactlimport{BulldozerTrick.h}
\kactlimport{RotatingCallipers.h}
\kactlimport{DualGraph.h}

%\kactlimport{CircleIntersection.h}
%\kactlimport{CircleTangents.h}
% \kactlimport{CircleLine.h}
%\kactlimport{CirclePolygonIntersection.h}
%\kactlimport{circumcircle.h}
\kactlimport{MinimumEnclosingCircle.h}
\kactlimport{UnionOfCircle.h}

\kactlimport{InsidePolygon.h}
%\kactlimport{PolygonArea.h}
%\kactlimport{PolygonCenter.h}
\kactlimport{PolygonCut.h}
%\kactlimport{PointInPolygon.h}
\kactlimport{PolygonUnion.h}
\kactlimport{ConvexHull.h}
%\kactlimport{HullDiameter.h}
\kactlimport{PointInsideHull.h}
%\kactlimport{LineHullIntersection.h}

%\kactlimport{ClosestPair.h}
\kactlimport{kdTree.h}
%\kactlimport{DelaunayTriangulation.h}

%\kactlimport{PolyhedronVolume.h}
%\kactlimport{Point3D.h}
%\kactlimport{3dHull.h}
%\kactlimport{sphericalDistance.h}
